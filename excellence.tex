%!TEX root = proposal.tex

%%% Important. To have correct table numberings
\renewcommand{\thetable}{\thesection\alph{table}}

\chapter[Excellence]{Excellence}
\label{cha:excellence}
\instructions{Your proposal must address a work programme topic for this call
for proposals.\\ \textit{This section of your proposal will be assessed only to
the extent that it is relevant to that topic.}}

\lipsum[1]


\section{Objectives}
\label{sec:objectives}

\instructions{
\begin{itemize}
\item Describe the specific objectives for the project\footnote{The term
`project' used in this template equates to an `action' in certain other Horizon
2020 documentation.}, which should be clear, measurable, realistic and
achievable within the duration of the project. Objectives should be consistent
with the expected exploitation and impact of the project (see section 2).
\end{itemize}}

\lipsum[1]


\section{Relation to the work programme}
\label{sec:relation-to-work-programme}
\instructions{
\begin{itemize}
\item Indicate the work programme topic to which your proposal relates, and
explain how your proposal addresses the specific challenge and scope of that
topic, as set out in the work programme.
\end{itemize}}

\lipsum[1]


\section{Concept and methodology}
\label{sec:conceptandmeth}

\subsection{Concept}
\label{sec:concept}

\instructions{
\begin{itemize}
\item Describe and explain the overall concept underpinning the project.
Describe the main ideas, models or assumptions involved. Identify any
inter-disciplinary considerations and, where relevant, use of stakeholder
knowledge. Where relevant, include measures taken for public/societal engagement
on issues related to the project.
Describe the positioning of the project, e.g., where it is situated in the
spectrum from `idea to application', or from `lab to market'. Refer to
Technology Readiness Levels where relevant. (See General Annex G of the work
programme);
\item Describe any national or international research and innovation activities
which will be linked with the project, especially where the outputs from these
will feed into the project;
\end{itemize}}

\lipsum[1]


\subsection{Methodology}
\label{sec:methodology}

\instructions{
\begin{itemize}
\item Describe and explain the overall methodology, distinguishing, as
appropriate, activities indicated in the relevant section of the work programme,
e.g., for research, demonstration, piloting, first market replication, etc;
\item Where relevant, describe how sex and/or gender analysis is taken into
account in the project's content.
\end{itemize} \textit{Please note that this question does not refer to gender
balance in the teams in charge of carrying out the project but to the content of
the planned research and innovation activities. Sex and gender analysis refers
to biological characteristics and social/cultural factors respectively. For
guidance on methods of sex / gender analysis and the issues to be taken into
account, please refer to
\url{http://ec.europa.eu/research/swafs/gendered-innovations/index_en.cfm?pg=home}}}

\lipsum[1]


\section{Ambition}
\label{sec:ambition}

\instructions{
\begin{itemize}
\item Describe the advance your proposal would provide beyond the state of the
art, and the extent the proposed work is ambitious.
\item Describe the innovation potential \textbf{(e.g., ground-breaking
objectives, novel concepts and approaches, new products, services or business
and organisational models)} which the proposal represents. Where relevant, refer
to products and services already available on the market. Please refer to the
results of any patent search carried out.
\end{itemize}}

\lipsum[1]
