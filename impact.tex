%!TEX root = proposal.tex

\chapter{Impact}
\label{cha:impact}

\section{Expected impacts}
\label{sec:expected-impact}

\instructions{
\emph{Please be specific, and provide only information that applies to the
proposal and its objectives. Wherever possible, use quantified indicators and
targets.}
\begin{itemize}
\item Describe how your project will contribute to:
\begin{itemize}
\item each of the expected impacts mentioned in the work programme, under the
relevant topic;
\item any substantial impacts not mentioned in the work programme, that would
enhance innovation capacity; create new market opportunities, strengthen
competitiveness and growth of companies, address issues related to climate
change or the environment, or  bring other important benefits for society
\end{itemize}
\item Describe any barriers/obstacles, and any framework conditions (such as
regulation, standards, public acceptance, workforce considerations, financing of
follow-up steps, cooperation of other links in the value chain), that may
determine whether and to what extent the expected impacts will be achieved.
(This should not include any risk factors concerning implementation, as covered
in section 3.2.)
\end{itemize} }

\lipsum[1]


\section{Measures to maximize impact}
\label{sec:maximize-impact}

\subsection{Dissemination and exploitation of results}
\label{sec:dissemination-exploitation}

\instructions{
\begin{itemize}
\item Provide a draft `\textbf{plan for the dissemination and
exploitation\footnote{See participant portal FAQ on how to address
\href{https://ec.europa.eu/research/participants/portal/desktop/en/support/faqs/faq-929.html}{dissemination
and exploitation} in Horizon 2020} of the project's results}'. Please note that
such a draft plan is an admissibility condition, unless the work programme topic
explicitly states that such a plan is not required.
\\
\\ Show how the proposed measures will help to achieve the expected impact of
the project.
\\
\\ The plan, should be proportionate to the scale of the project, and should
contain measures to be implemented both during and after the end of the project.
For innovation actions, in particular, please describe a credible path to
deliver these innovations to the market.
\end{itemize}

\textit{Your plan for the dissemination and exploitation of the project's
results is key to maximizing their \textbf{impact}. This plan should describe,
in a concrete and comprehensive manner, the \textbf{area} in which you expect to
make an impact and \textbf{who} are the potential users of your results. Your
plan should also describe \textbf{how} you intend to use the appropriate
channels of dissemination and interaction with potential users.}

\textit{Consider the full range of potential users and uses, including research,
commercial, investment, social, environmental, policy-making, setting standards,
skills and educational training where relevant.}

\textit{Your plan should give due consideration to the possible
\textbf{follow-up} of your project, once it is finished. Its exploitation could
require additional investments, wider testing or scaling up. Its exploitation
could also require other pre-conditions like regulation to be adapted, or value
chains to adopt the results, or the public at large being receptive to your
results. }

\begin{itemize}
\item Include a business plan where relevant.
\item As relevant, include information on how the participants will manage the
research data generated and/or collected during the project, in particular
addressing the following issues:
\begin{itemize}
\item What types of data will the project generate/collect?
\item What standards will be used?
\item How will this data be exploited and/or shared/made accessible for
verification and re-use? If data cannot be made available, explain why.
\item How will this data be curated and preserved?
\item  How will the costs for data curation and preservation be covered?
\end{itemize}
\end{itemize}

\textit{Actions under Horizon 2020 participate in the `Pilot on Open Research
Data in Horizon 2020, except if indicated otherwise (`opt out'.)\footnote{Opting
out of the Open Research Data Pilot is possible, both before and after the grant
signature. For further guidance on open research data and data management,
please refer to the
\href{http://ec.europa.eu/research/participants/docs/h2020-funding-guide/index_en.htm}{H2020
Online Manual} on the Participant Portal.}. Once the action has started
(\textbf{not} at application stage) those beneficiaries which do not opt-out, will
need to create a more detailed Data Management Plan for making their data
findable, accessible, interoperable and reusable (FAIR).}

\textit{You will need an appropriate consortium agreement to manage (among other
things) the ownership and access to key knowledge (IPR, research data, etc.).
Where relevant, these will allow you, collectively and individually, to pursue
market opportunities arising from the project's results.}

\textit{The appropriate structure of the consortium to support exploitation is
addressed in section 3.3.}

\begin{itemize}
\item Outline the strategy for \textbf{knowledge management and protection}.
Include measures to provide \textbf{open access} (free on-line access, such as
the `greenz' or `gold' model) to peer-reviewed scientific publications which
might result from the project\footnote{Open Access must be granted to all
scientific publications resulting from Horizon 2020 actions (in particular
scientific peer reviewed articles). Further guidance on Open Access is available
in the
\href{http://ec.europa.eu/research/participants/docs/h2020-funding-guide/index_en.htm}{H2020
Online Manual} on the Participant Portal.}.
\end{itemize}

\textit{Open Access publishing (also called `gold' open access) means that an
article is immediately provided in open access mode by the scientific publisher.
The associated costs are usually shifted \textbf{away from readers, and instead
(for example) to the university or research institute to which the} researcher
is affiliated, or to the funding agency supporting the research. Gold open
access costs are fully eligible as part of the grant. Note that if the gold
route is chosen, a copy of the publication has to be deposited in a repository
as well.}

\textit{Self-archiving (also called `green' open access) means that the
published article or the final peer-reviewed manuscript is archived by the
researcher---or a representative---in an online repository before, after or
alongside its publication. Access to this article is often---but not
necessarily---delayed (`embargo period'), as some scientific publishers may wish
to recoup their investment by selling subscriptions and charging
pay-per-download/view fees during an exclusivity period. } }

\lipsum[1]


\subsection[Communication activities]{Communication
activities\instructions{\footnote{See participant portal FAQ on how to address
\href{https://ec.europa.eu/research/participants/portal/desktop/en/support/faqs/faq-930.html}{communication
activities} in Horizon 2020}\footnote{For further guidance on communicating EU
research and innovation for project participants, please refer to the
\href{http://ec.europa.eu/research/participants/docs/h2020-funding-guide/index_en.htm}{H2020
Online Manual} on the Participant Portal.}}}
\label{sec:communication}

\instructions{
\begin{itemize}
\item Describe the proposed communication measures for promoting the project and
its findings during the period of the grant. Measures
should be proportionate to the scale of the project, with clear objectives.
They should be tailored to the needs of different target audiences, including
groups beyond the project's own community. Where relevant, include measures for
public/societal engagement on issues related to the project.
\end{itemize} }

\lipsum[1]
